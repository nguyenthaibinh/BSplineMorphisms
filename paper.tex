%\documentclass[preprint,authoryear,review,12pt]{elsarticle}
\documentclass{frontiersSCNS}

%% Use the option review to obtain double line spacing
%% \documentclass[preprint,review,12pt]{elsarticle}

%% Use the options 1p,two column; 3p; 3p,twocolumn; 5p; or 5p,twocolumn
%% for a journal layout:
%% \documentclass[final,1p,times]{elsarticle}
%% \documentclass[final,1p,times,twocolumn]{elsarticle}
%% \documentclass[final,3p,times]{elsarticle}
%% \documentclass[final,3p,times,twocolumn]{elsarticle}
%% \documentclass[final,5p,times]{elsarticle}
%% \documentclass[final,5p,times,twocolumn]{elsarticle}


\usepackage{color}
\usepackage{multirow,booktabs,ctable,array}
\usepackage{lscape}
\usepackage{amsmath}
\usepackage{lineno}
\usepackage{ulem}
\usepackage{setspace}
\usepackage{listings}
\usepackage{float}


\floatstyle{plain}
\newfloat{command}{thp}{lop}
\floatname{command}{Command}

%\usepackage[nomarkers,notablist]{endfloat}

%% if you use PostScript figures in your article
%% use the graphics package for simple commands
%% \usepackage{graphics}
%% or use the graphicx package for more complicated commands
%% \usepackage{graphicx}
%% or use the epsfig package if you prefer to use the old commands
%% \usepackage{epsfig}

%% The amssymb package provides various useful mathematical symbols
\usepackage{amssymb}
%% The amsthm package provides extended theorem environments
% \usepackage{amsthm}
 
 \usepackage{makecell}

%% The lineno packages adds line numbers. Start line numbering with
%% \begin{linenumbers}, end it with \end{linenumbers}. Or switch it on
%% for the whole article with \linenumbers after \end{frontmatter}.
%% \usepackage{lineno}

%% natbib.sty is loaded by default. However, natbib options can be
%% provided with \biboptions{...} command. Following options are
%% valid:

%%   round  -  round parentheses are used (default)
%%   square -  square brackets are used   [option]
%%   curly  -  curly braces are used      {option}
%%   angle  -  angle brackets are used    <option>
%%   semicolon  -  multiple citations separated by semi-colon
%%   colon  - same as semicolon, an earlier confusion
%%   comma  -  separated by comma
%%   numbers-  selects numerical citations
%%   super  -  numerical citations as superscripts
%%   sort   -  sorts multiple citations according to order in ref. list
%%   sort&compress   -  like sort, but also compresses numerical citations
%%   compress - compresses without sorting
%%
%% \biboptions{comma,round}

% \biboptions{}

\providecommand{\OO}[1]{\operatorname{O}\bigl(#1\bigr)}

\graphicspath{{./Figures/}
                          }

\long\def\symbolfootnote[#1]#2{\begingroup%
\def\thefootnote{\fnsymbol{footnote}}\footnote[#1]{#2}\endgroup}

    \usepackage{color}

    \definecolor{listcomment}{rgb}{0.0,0.5,0.0}
    \definecolor{listkeyword}{rgb}{0.0,0.0,0.5}
    \definecolor{listnumbers}{gray}{0.65}
    \definecolor{listlightgray}{gray}{0.955}
    \definecolor{listwhite}{gray}{1.0}

\newcommand{\lstsetcpp}
{
\lstset{frame = tb,
        framerule = 0.25pt,
        float,
        fontadjust,
        backgroundcolor={\color{listlightgray}},
        basicstyle = {\ttfamily\scriptsize},
        keywordstyle = {\ttfamily\color{listkeyword}\textbf},
        identifierstyle = {\ttfamily},
        commentstyle = {\ttfamily\color{listcomment}\textit},
        stringstyle = {\ttfamily},
        showstringspaces = false,
        showtabs = false,
        numbers = none,
        numbersep = 6pt,
        numberstyle={\ttfamily\color{listnumbers}},
        tabsize = 2,
        language=[ANSI]C++,
        floatplacement=!h,
        caption={},
        captionpos=b,
        }
}


\copyrightyear{}
\pubyear{}

\def\journal{Neuroscience}%%% write here for which journal %%%
\def\DOI{}
\def\articleType{Methods}
\def\keyFont{\fontsize{8}{11}\helveticabold }
\def\firstAuthorLast{Tustison {et~al.}} %use et al only if is more than 1 author
\def\Authors{Nicholas J. Tustison\,$^{1,*}$ and Brian A. Avants\,$^{2}$
 }
% Affiliations should be keyed to the author's name with superscript numbers and be listed as follows: Laboratory, Institute, Department, Organization, City, State abbreviation (USA, Canada, Australia), and Country (without detailed address information such as city zip codes or street names).
% If one of the authors has a change of address, list the new address below the correspondence details using a superscript symbol and use the same symbol to indicate the author in the author list.
\def\Address{$^{1}$University of Virginia, Department of Radiology and Medical Imaging, Charlottesville, VA, USA \\
$^{2}$Penn Image Computing and Science Laboratory, University of Pennsylvania, Department of Radiology, Philadelphia, PA, USA }
% The Corresponding Author should be marked with an asterisk
% Provide the exact contact address (this time including street name and city zip code) and email of the corresponding author
\def\corrAuthor{Nick Tustison}
\def\corrAddress{University of Virginia, Department of Radiology and Medical Imaging,  480 Ray C Hunt Drive, Charlottesville, VA, 22903}
\def\corrEmail{ntustison@virginia.edu}

% \color{FrontiersColor} Is the color used in the Journal name, in the title, and the names of the sections


\begin{document}
\onecolumn
\firstpage{1}

\title[Explicit B-spline Regularization]{Explicit B-spline Regularization in Diffeomorphic Image Registration}
\author[\firstAuthorLast ]{\Authors}
\address{}
\correspondance{}
\editor{}
\topic{}

\maketitle

%\linenumbers


\begin{abstract}
Important methodological developments in the evolution of image registration algorithms include those 
in which the correspondence relationship is characterized by diffeomorphisms.
The  popularity of these approaches is due largely to their 
topological properties and success in providing biologically plausible 
solutions to small and large deformation 
estimation problems. Variants of diffeomorphic algorithms include those characterized by 
time-varying and constant velocity fields, and symmetrical considerations. 
Prior information in the form of regularization is used to enforce transform plausibility taking the form of physics-based constraints or through some approximation thereof, e.g. Gaussian smoothing (a la Thirion's Demons \citep{thirion1998}).  In the context of the original Demons' framework, the traditional free-form deformation method can be viewed as a variant in which explicit regularization is achieved through the B-spline basis functions.  
This characterization can be used to provide alternative ``flavors'' of diffeomorphic image registration solutions with several advantages which we describe in this work.  Implementation is open source and available through the Insight Toolkit and our Advanced Normalization Tools (ANTs) repository.  A thorough evaluation with the well-known SyN algorithm \citep{avants2008}, implemented within the same framework, and its B-spline analog is performed using open data and open source evaluation tools.
\tiny
\keyFont{ \section{Keywords:} 
open science best practices comparison reproducibility software }
\end{abstract}







%\begin{itemize}
%  \item We should trace the use of explicit regularization from 
%  \item The advantage is not the parameterization per se (contra Tom) but the other 
%  benefits listed (Vercuraten, Non-parametric Diffeomorphic Image Registration with the Demons Algorithm).
%  \item This paper provides the link between traditional B-spline approaches and other
%  approaches, i.e. Gaussian smoothing, Demons, 
%  \item The ability to weight the boundaries provides a natural way of enforcing Dirichlet boundary conditions which is consistent with the geometric (i.e. B-spline) modeling. (Cahill, SPIE 2012).
%  \item Fluid vs. Elastic B-spline registration algorithms
%  \item Fitting a continuous object (as opposed to the discrete gaussian which has no
%        continuity nor does convolution using FFT provide continuity constraints).
%  \item Fitting routine is parallelized for fast sampling as is sampling to go from continuous object to sampled b-spline object.
%  \item Everything is open-source in ANTs and ITK.
%  \item Also, we should talk about how B-spline SyN works really well for large smoothing  but Gaussian SyN does not.
%\end{itemize}
%
%
%Also talk about fixed boundary conditions in the context of Nathan Cahill, SPIE 2012.
%




%% MSC codes here, in the form: \MSC code \sep code
%% or \MSC[2008] code \sep code (2000 is the default)

%%
%% Start line numbering here if you want
%%
% \linenumbers

%% main text

\section{Introduction}
%State the objectives of the work and provide an adequate background, avoiding a detailed literature survey or a summary of the results.

Establishment of anatomical and functional correspondence
is a crucial step towards insight into certain biological 
sciences.  Neuroscience research efforts, such as characterizing 
brain morphology, require accurate and robust methods for
producing such mappings.  The 
extensive literature detailing methodology is evidence of the rich history of 
algorithmic development which continues contemporaneously.
We highlight several key contributions which are particularly relevant to the work presented.

Free-form deformation (FFD) image registration, preceded by related work for
geometric modeling \citep{sederberg1986}, originated with such important
contributions as \cite{szeliski1997,thevenaz1998,rueckert1999}.  Continued 
development within this early spline-based paradigm produced additional innovations
such as integrated similarity metrics \citep[e.g.][]{mattes2003}, additional transformation
constraints \citep[e.g.][]{rohlfing2003}, and open source contributions \citep[e.g.][]{ibanez2005,klein2010,shackleford2010}.

Parallel to this branch of algorithmic progress are the informally 
denoted ``dense transforms''
perhaps best exemplified by Thirion's seminal contribution \citep{thirion1998}.
Relationships with earlier elastic \citep{bajcsy1989,gee1993} and fluid \citep{christensen1996} registration methods are detailed in 
the works of \cite{bro-nielsen1996} and \cite{pennec1999} who observe that
Gaussian smoothing, characteristic of Demons, of the update or displacement
field is a greedy approximation for solving the partial differential equations governing
the physics of an elastic or fluid deformation.  However, the use of such 
approximations comes with the caveat that physical properties, such as topological
regularity, are no longer guaranteed.
 
It is interesting to note that within this context, traditional FFD algorithms
can be viewed as a type of fluid-like Demons approach 
where, rather than produced by Gaussian convolution, smooth fields are produced using B-splines for roughly approximating the solution to the viscous fluid system.  This was hinted at in our earlier work \citep{tustison2009} where we showed that fitting the update field to a B-spline object using a fast approximation routine \citep{tustison2006} is equivalent to a preconditioning of the standard gradient used in gradient descent-based FFD optimization.  This preconditioning used to reduce the 

Further details can be gleaned from the ITK implementation of this work%
\footnote{
http://www.itk.org/Doxygen/html/classitk\_1\_1BSplineSmoothingOnUpdateDisplacementFieldTransform.html
}
which permits both B-spline smoothing on the update (``viscous'') and total (``elastic'') 
displacement fields at each iteration (cf
analogous Gaussian, i.e. Demons, implementation%
\footnote{
http://www.itk.org/Doxygen/html/classitk\_1\_1GaussianSmoothingOnUpdateDisplacementFieldTransform.html
}).

Continuing from the work of \cite{christensen1996} and subsequent exploration into the mathematical formalisms of diffeomorphisms \citep[e.g.][]{dupuis1998}, 
the well-known Large Deformation Diffeomorphic Metric Mapping (LDDMM) algorithm  
was proposed in \cite{Beg2005}.  In contrast to the mapping produced
by \cite{christensen1996}, LDDMM yields the geodesic solution in the space of diffeomorphisms between two images.  



%Other important image registration research
%reflected increased emphasis on topological transformation considerations
%in modeling biological/physical systems where topology is 
%consistent throughout the course of deformation or a 
%homeomorphic relationship is assumed between image domains.
%Methods such as LDDMM \citep{beg2005} optimize time-varying velocity field 
%flows to yield diffeomorphic transformations. 


Alternatively, the FFD 
variant reported in \cite{rueckert2006} enforced diffeomorphic transforms
by concatenating multiple FFD transforms, each of which is constrained
to describe a one-to-one mapping.  Another FFD registration
incorporated the recent log-Euclidean framework for enforcing diffeomorphic
transformations \cite{Modat2011}.
Recently, the work of 
\cite{de-craene2011} combined these registration concepts into a single
framework called {\em temporal free-form deformation} in which the 
time-varying velocity field characteristic of LDDMM-style algorithms
is modeled using a 4-D B-spline object (3-D + time).  Integration of 
the velocity field yields the mapping between parameterized time points.






%
%
%framework developed independently of Demons and LDDMM/DARTEL.
%But then the FFD people saw how useful the diff. properties were and
%tried to integrate B-splines into the diff. framework.
%
%
%The early deformable work of 
%\begin{itemize}
%  \item xx Early deformable work \cite{broit1981,bajcsy1989}.
%  \item FFD \cite{Sleziski,Rueckert1989}
%  \item Christensen's fluid work
%  \item Optical flow?
%  \item Fast fluid registration of medical images (Bro-Nielsen and Gramkow).
%  \item Consistent image registration
%  \item Demons and Diff demons
%  \item LDDMM 
%  \item DARTEL
%  \item Arno Klein's study
%\end{itemize}
%
%
%
%Early work included the elastic deformation


%Significant algorithmic developments characterizing modern intensity-based
%image registration research include the B-spline parameterized approach
%(so called {\em free-form deformation}) with early contributions 
%including \cite{szeliski1997,thevenaz1998,rueckert1999}.  Amongst the 
%variant extensions, the 
%{\em directly manipulated free-form deformation} approach  \cite{tustison2009} 
% addressed the hemstitching issue associated with steepest descent traversal of
%problematic energy topographies during the course of optimization.
%%caused by the distribution of the uniform 
%%B-spline shape functions over the transformation domain 
%
%
%In this work, we describe our extension to these methods.  Similar to 
%\cite{de-craene2011}, we also use an $N$-D + time B-spline object to 
%represent the characteristic velocity field.  However, we use the 
%directly manipulated free-form deformation optimization formulation to improve 
%convergence during the course of optimization.  This also facilitates
%modeling temporal periodicity and
%the enforcement of stationary boundaries consistent with diffeomorphic
%transforms.
%We also incorporate B-spline mesh multi-resolution capabilities
%for increased control during registration progression.  
%Most importantly, we also describe
%the parallelized algorithmic implementation as open source available through the Insight Toolkit.%
%\footnote{
%http://www.itk.org/
%}
%
%We first describe the methodology by laying out a mathematical description 
%of the various algorithmic elements coupled with implementation details
%where appropriate.  This is followed by an evaluation on publicly available
%brain data. 


\section{Material and Methods}
%Provide sufficient detail to allow the work to be reproduced. Methods already published should be indicated by a reference: only relevant modifications should be described.

Three popular algorithms [1,2,3] (and their B-spline analogs) were implemented as part of the recent refactoring of the Insight Toolkit (ITK) [5] and included as part of the Advanced Normalization Tools (ANTs) repository [6].  Thus, all code used for this work is available as open source.   For comparison, the ITK-based, well-vetted SyN algorithm [3], which uses explicit Gaussian regularization, was used to provide the canonical mappings in the MICCAI 2012 multi-atlas labeling challenge [7] between 20 training subjects and 15 testing subjects (140 labeled structures).  Using the exact same parameters, except those specific to the regularization, we produced the analogous B-spline SyN mappings for the same data and compared transformations using the Dice overlap.

\subsection{Public Data}

\subsection{Implementation}

\section{Results}
%Results should be clear and concise.


\section{Discussion and Conclusions}
%This should explore the significance of the results of the work, not repeat them. A combined Results and Discussion section is often appropriate. Avoid extensive citations and discussion of published literature.

B-spline regularization is easily adapted into the diffeomorphic registration framework and performs comparably to analogous algorithms which we demonstrated in the case of SyN.  Even more importantly, all source code and data used in this work is publicly available for the interested researcher to explore.

%\section{Conclusions}
%The main conclusions of the study may be presented in a short Conclusions section, which may stand alone or form a subsection of a Discussion or Results and Discussion section.

%% The Appendices part is started with the command \appendix;
%% appendix sections are then done as normal sections
%% \appendix

%% \section{}
%% \label{}

%% References
%%
%% Following citation commands can be used in the body text:
%% Usage of \cite is as follows:
%%   \citep{key}          ==>>  [#]
%%   \cite[chap. 2]{key} ==>>  [#, chap. 2]
%%   \citet{key}         ==>>  Author [#]

%% References with bibTeX database:

\bibliographystyle{frontiersinSCNS}
%\bibliographystyle{plain}
\bibliography{references}


%% Authors are advised to submit their bibtex database files. They are
%% requested to list a bibtex style file in the manuscript if they do
%% not want to use model1-num-names.bst.

%% References without bibTeX database:

% \begin{thebibliography}{00}

%% \bibitem must have the following form:
%%   \bibitem{key}...
%%

% \bibitem{}

% \end{thebibliography}


\end{document}

%%
%% End of file `elsarticle-template-1-num.tex'.
